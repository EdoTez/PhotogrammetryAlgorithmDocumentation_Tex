% Esempio per lo stile supsi
\documentclass[twoside]{supsistudent} 

% per settare noindent
\setlength{\parindent}{0pt}


% Crea un capitolo senza numerazione che pero` appare nell'indice %
\newcommand{\problemchapter}[1]{%
  \chapter*{#1}%
  \addcontentsline{toc}{chapter}{#1}%
\markboth{#1}{#1}
}

% Numerazione delle appendici secondo norma
\addto\appendix{
\renewcommand{\thesection}{\Alph{chapter}.\arabic{section}}
\renewcommand{\thesubsection}{\thesection.\arabic{subsection}}}

\setcounter{secnumdepth}{5} 	%per avere più livelli nei titoli
\setcounter{tocdepth}{5}		%per avere più livelli nell'indice


\titolo{Nuovo algoritmo fotogrammetria - Azienda esterna REX SA}
\studente{Terzi Edoardo}
\relatore{Mattei Yari}
\correlatore{-}
\committente{REX SA}
\corso{Ingegneria informatica (TP)}
\modulo{C10051 Progetto di diploma}
\anno{2019}



\begin{document}

\pagenumbering{alph}
\maketitle
\onehalfspacing
\frontmatter


\pagenumbering{roman}
\tableofcontents
% \listoffigures
% \listoftables	

\mainmatter % suddivide in capitolo 1,2,3 ecc.
\pagenumbering{arabic}
\setcounter{page}{1}

\chapter{Abstract}
Questo progetto nasce dalla necessità dell'azienda REX di Mendrisio di creare un sistema per la progettazione di una 
passerella posta fra i binari di un passo ferroviario. L'azienda in questine si occupa della produzione, della 
progettazione e della vendita di molteplici articoli tecnici tra i quali le passerelle. \\

Inizialmente il sistema, costituito da una piattaforma WEB, si occupava di ricevere in input un file DXF realizzato da un geometra 
dell'azienda. Questo file doveva contenere le rilevazioni del geometra, il calcolo delle griglie che costiutiscono la passerella e 
il disegno di queste. Infine la piattaforma generava un documento PDF contenente le istruzioni specifiche per il montaggio della passerella.
Inoltre era possibile gestire i clieniti e gli ordini effettuati mediante la piattaforma. \\

Attualmente, in seguito a un progetto di semestre svloto da due studenti della SUPSI, la piattaforma è migliorata mantenendo sempre 
l'aspetto relativo alla gestione degli ordini, dei clienti e del PDF ma viene alleggerito il lavoro del geometra. Quest'ultimo deve inserire 
nel file DXF solamente le misure effettuate e non deve calcolare a mano le griglie necessarie alla costruzione della passerella. \\

L'obiettivo di questo progetto di diploma è quello di rendere ancora più semplice ed automatico l'utilizzo dell'intero sistema mediante l'utilizzo 
di tecniche di processamento dell'immagine. In particolare si vuole realizzare un algoritmo che sia in grado di rilevare le misure necessarie alla costruzione
della passerella partendo da una semplice foto della scena, quindi del passo ferroviario dove si vuole installare la passerella. \\

\chapter{Progetto}

\section{Descrizione}
Rex SA è una società manifatturiera: uno dei molti prodotti commercializzati con il proprio marchio è relativo alla
produzione di passaggi di servizio ferroviari in resina per l’attraversamento nelle stazioni ferroviarie e per uscite di
sicurezza nelle gallerie di treno o metropolitana, articoli commercializzati con il marchio SWISSCROSS GFK
La sfida del progetto risiede nell’implementazione di un sistema capace di creare e visualizzare in modalità aumentata
(sopra una fotografia) un modello virtuale di un passaggio GFK partendo da un'immagine e attivando automaticamente i
processi di produzione, consegna, installazione e, in seguito, di manutenzione del prodotto.
Grazie alla soluzione proposta, REX sarà in grado rafforzare la propria posizione sul mercato svizzero e modificare il
proprio modello di business per iniziare una vendita di servizi digitali a livello internazionale.
Con il progetto si vuole incrementare la competitività dell’azienda delegando al cliente la progettazione e la visione virtuale
del risultato finale in modalità “self-service” prima ancora di confermare l’ordine tramite la piattaforma cloud,
automatizzando quindi ordinazione, produzione e delivery in un vero e proprio processo di Industria 4.0

\section{Obiettivi}
L'obiettivo principale è quello di sviluppare un algoritmo in grado di sfruttare le metriche conosciute (scarto binari e modelli
3D di binari, traversine e viterie) per ricavare una matrice omografica in grado di calcolare il modello completo di passerella
sulla base di una immagine.
Allineamento e sovrapposizione dei modelli generati da più immagini verranno realizzati sfruttando appositi marcatori
posizionati sul terreno.
Obiettivo di questo algoritmo è quello di permettere l'estrazione dei punti caratteristici da un immagine (features) necessari
al calcolo della forma e delle misure di una passerella GFK.

\section{Compiti}
Sviluppo di un algoritmo in grado di elaborare un immagine fotografica e ricavarne la relativa matrice omografa.
Tramite questa matrice si potrà "addrizzare" l'immmagine e quindi calcolare le dimensioni reali della passaggio pedonale
che dovrà quindi essere visualizzato in modalità "realtà aumentata" sopra la fotografia

\chapter{Introduzione}

\section{Premessa}

\section{Contesto motivazionale}

\section{Scopo}

\chapter{Pianificazione}

\section{Approccio iniziale}

\section{Svlogimento del lavoro}

\section{Diagramma di Gantt}

\chapter{Analisi}

\section{Requisiti}

\section{Architettura}

\section{Tecnologie}

\chapter{Diagrammi}

\section{Casi d'uso}

\section{Sequenza}

\chapter{Implementazione}

\section{Linguaggi, librerie e framework}
\subsection{Java}
\subsection{Python}
\subsection{OpenCV}

\section{Identificazione punti e binari}

\section{Calibratore}

\section{Matrice Omografica}

\section{Interfacciamento sistema pre esistente}

\section{Gestione errori}

\section{Testing}

\chapter{Conclusioni}

\section{Problematiche}

\section{Sviluppi futuri}

\section{Considerazioni finali}

% \lipsum[23]
% Esempio di citazione \cite{4538384}, \cite{5357331,4523385}, \cite{1705631}.
% \footnote{Questa \`e una nota a pi\'e di pagina.}
% \footnote{Questa \`e un'altra nota a pi\'e di pagina.}
% \lipsum[23]

% \subsection{Sotto sezione}

% \texttt{Questo testo ha una spaziatura fissa}

% \textit{Questo testo \`e in italico}

% \textbf{Questo testo \`e in grassetto}

% \textsc{Questo testo \`e in maiuscoletto}

% \underline{Questo testo è sottolineato}

% Citazione:
% \begin{quote}
% \lipsum[23]
% \end{quote}

% \chapter{Titolazione}

% \lipsum[13]

% \begin{itemize}
%   \item Elemento A
%   \item Elemento B
%   \item Elemento C
% \end{itemize}

%\begin{itemize}
%  \item[-] Elemento A
%  \item[-] Elemento B
%  \item[-] Elemento C
%\end{itemize}
%
%\begin{enumerate}
%  \item Alpha
%  \item Beta
%  \item Gamma
%\end{enumerate}

% \lipsum[23]
%\section{Sezione}
%
%\lipsum[23]
%
%\subsection{Sotto sezione}
%
%Un po' di matematica: \newline
%
%\begin{math}
%\frac{n!}{k!(n-k)!} = {n \choose k}
%\end{math} \newline
%
%Un po' di matematica centrata:
%
%\begin{center}
%\begin{math}
%\frac{n!}{k!(n-k)!} = {n \choose k}
%\end{math}
%\end{center}
%
%Oppure con \$\$
%
%$$
%\frac{n!}{k!(n-k)!} = {n \choose k}
%$$
%
%Oppure anche direttamente nel testo ${1}\over{n}$ \\
%
%\lipsum[23]

\bibliographystyle{unsrt}
\bibliography{bibliografia}
\end{document}
